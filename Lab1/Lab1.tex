%--------------------------INITIALISATION DU DOCUMENT---------------------%
%											%														%											%
%                        >>>> Ne pas modifier cette partie <<<<			%
																																					
\documentclass[letterpaper,12pt,oneside,final,fleqn]{book}
\input{packages}

\usepackage{multicol}
\usepackage{graphicx}
\usepackage{wrapfig}
\graphicspath{ {images/} }
\setlength{\columnsep}{1cm}

\begin{document}

%========================= Page de couverture ============================%

\newcommand\prenomUn{Alice}	
\newcommand\nomUn{Breault-g}
\newcommand\matriculeUn{1836459}
\newcommand\prenomDeux{Thanina}	
\newcommand\nomDeux{Haddad}
\newcommand\matriculeDeux{1839982}
\newcommand\dateRemise{09 février 2017}
\newcommand\groupe{04}		
%\newgeometry{tmargin=2.0cm, bmargin=2.0cm, lmargin=2.25cm, rmargin=2.25cm, headsep=1.0cm}
\newgeometry{top=2cm}
\definecolor{gris1}{gray}{0.75}

\newcommand{\encadre}[1]{
\setlength\fboxsep{5mm}\setlength\fboxrule{1pt}
\begin{center}
\fcolorbox{black}{gris1}{
\begin{minipage}{0.94\textwidth}{#1}\end{minipage}}
\end{center}}

% encadre blanc
\newcommand{\boite}[1]{
\setlength\fboxsep{5mm}\setlength\fboxrule{1pt}
\begin{center}
\fcolorbox{black}{white}{
\begin{minipage}{0.5\textwidth}{#1}\end{minipage}}
\end{center}}


%\begin{document}

\thispagestyle{empty}

\encadre{
\begin{center}
\bf
{\Large \scshape 
\vspace{10mm}
Polytechnique Montr\'eal
\\
D\'epartement de G\'enie Physique
}
\\
{\Huge
\

PHS1102 \\
Champs électromagnétiques
\\
Hiver 2017

\

Laboratoire 1:\\
Mesure de la permitivité

\vspace{15mm}

}
\end{center}
}

\vfill

\fcolorbox{black}{white}{
\begin{minipage}{0.94\linewidth}

\vspace{4mm}

\begin{multicols}{3}
	{\bf Nom: }\nomUn
	\\
	\\
	{\bf Nom: }\nomDeux 

	{\bf Pr\'enom: }\prenomUn 
	\\
	\\
	{\bf Pr\'enom: }\prenomDeux
 
	{\bf Matricule: }\matriculeUn
	\\
	\\
	{\bf Matricule: }\matriculeDeux
\end{multicols}

\vspace{2mm}

{\bf Date de remise: }\dateRemise \hspace{38mm} {\bf Section: }\groupe

\vspace{4mm}

\end{minipage}}

\vfill

\restoregeometry
%\end{document}

%========================= Introduction ============================%

\newpage \section*{Table des matières}



\newpage \section*{Introduction}

La réalisation de ce laboratoire a pour objectifs la comparaison de l'efficacité de deux techniques de mesure de la capacité d'un condensateur ainsi que la détermination de trois diélectriques en fonction de leur permittivité relative.

\newpage \section*{Manipulations}

\end{document}
# PHS1102
