%--------------------------INITIALISATION DU DOCUMENT---------------------%
%											%														%											%
%                        >>>> Ne pas modifier cette partie <<<<			%
																																					
\documentclass[letterpaper,12pt,oneside,final,fleqn]{book}



%%
%%  Version: 2014-10-28
%%
%%  Accepte les caractères accentués dans le document (UTF-8).
\usepackage[utf8]{inputenc}
%%
%% Support pour l'anglais et le français (français par défaut).
%\usepackage[cyr]{aeguill}
\usepackage{lmodern}      % Police de caractères plus complète et généralement indistinguable visuellement de la police standard de LaTeX (Computer Modern).
\usepackage[T1]{fontenc}  % Bon encodage des caractères pour qu'Acrobat Reader reconnaisse les accents et les ligatures telles que ffi.
\usepackage[english,frenchb]{babel} % le langage par défaut est le dernier de la liste, c'est-à-dire français
%%
%% Charge le module d'affichage graphique.
\usepackage{graphicx}
\usepackage{epstopdf}  % Permet d'utiliser des .eps avec pdfLaTeX.
%%
%% Recherche des images dans les répertoires.
\graphicspath{{./images/}{./dia/}{./gnuplot/}}
%%
%% Un float peut apparaître seulement après sa définition, jamais avant.
\usepackage{flafter,placeins}
%%
%% Utilisation de natbib pour les citations et la bibliographie.
\usepackage{natbib}
%%
%% Autres packages.
\usepackage{amsmath,color,soulutf8,longtable,colortbl,setspace,ifthen,xspace,url,pdflscape,tikz,pgfplots}
%%
%% Support des acronymes.
\usepackage[nolist]{acronym}
\onehalfspacing                % Interligne 1.5.
%%
%% Définition d'un style de page avec seulement le numéro de page à
%% droite. On s'assure aussi que le style de page par défaut soit
%% d'afficher le numéro de page en haut à droite.
\usepackage{fancyhdr}
\fancypagestyle{pagenumber}{\fancyhf{}\fancyhead[R]{\thepage}}
\renewcommand\headrulewidth{0pt}
\makeatletter
\let\ps@plain=\ps@pagenumber
\makeatother
%%
%% Module qui permet la création des bookmarks dans un fichier PDF.
%\usepackage[dvipdfm]{hyperref}
\usepackage{hyperref}
\usepackage{caption}  % Hyperlien vers la figure plutôt que son titre.

\usepackage{esint}
\usepackage{geometry}
\usepackage{enumerate}



\usepackage{multicol}
\usepackage{graphicx}
\usepackage{wrapfig}
\graphicspath{ {images/} }
\setlength{\columnsep}{1cm}

\begin{document}

%========================= Page de couverture ============================%

\newcommand\prenomUn{Alice}	
\newcommand\nomUn{Breault-g}
\newcommand\matriculeUn{1836459}
\newcommand\prenomDeux{Thanina}	
\newcommand\nomDeux{Haddad}
\newcommand\matriculeDeux{1839982}
\newcommand\dateRemise{09 février 2017}
\newcommand\groupe{04}		
%\newgeometry{tmargin=2.0cm, bmargin=2.0cm, lmargin=2.25cm, rmargin=2.25cm, headsep=1.0cm}
\newgeometry{top=2cm}
\definecolor{gris1}{gray}{0.75}

\newcommand{\encadre}[1]{
\setlength\fboxsep{5mm}\setlength\fboxrule{1pt}
\begin{center}
\fcolorbox{black}{gris1}{
\begin{minipage}{0.94\textwidth}{#1}\end{minipage}}
\end{center}}

% encadre blanc
\newcommand{\boite}[1]{
\setlength\fboxsep{5mm}\setlength\fboxrule{1pt}
\begin{center}
\fcolorbox{black}{white}{
\begin{minipage}{0.5\textwidth}{#1}\end{minipage}}
\end{center}}


%\begin{document}

\thispagestyle{empty}

\encadre{
\begin{center}
\bf
{\Large \scshape 
\vspace{10mm}
Polytechnique Montr\'eal
\\
D\'epartement de G\'enie Physique
}
\\
{\Huge
\

PHS1102 \\
Champs électromagnétiques
\\
Hiver 2017

\

Laboratoire 1:\\
Mesure de la permitivité

\vspace{15mm}

}
\end{center}
}

\vfill

\fcolorbox{black}{white}{
\begin{minipage}{0.94\linewidth}

\vspace{4mm}

\begin{multicols}{3}
	{\bf Nom: }\nomUn
	\\
	\\
	{\bf Nom: }\nomDeux 

	{\bf Pr\'enom: }\prenomUn 
	\\
	\\
	{\bf Pr\'enom: }\prenomDeux
 
	{\bf Matricule: }\matriculeUn
	\\
	\\
	{\bf Matricule: }\matriculeDeux
\end{multicols}

\vspace{2mm}

{\bf Date de remise: }\dateRemise \hspace{38mm} {\bf Section: }\groupe

\vspace{4mm}

\end{minipage}}

\vfill

\restoregeometry
%\end{document}

%========================= Introduction ============================%

\newpage \section*{Table des matières}



\newpage \section*{Introduction}

La réalisation de ce laboratoire a pour objectifs la comparaison de l'efficacité de deux techniques de mesure de la capacité d'un condensateur ainsi que la détermination de trois diélectriques en fonction de leur permittivité relative.

\newpage \section*{Manipulations}

\end{document}
# PHS1102
