


%%
%%  Version: 2014-10-28
%%
%%  Accepte les caractères accentués dans le document (UTF-8).
\usepackage[utf8]{inputenc}
%%
%% Support pour l'anglais et le français (français par défaut).
%\usepackage[cyr]{aeguill}
\usepackage{lmodern}      % Police de caractères plus complète et généralement indistinguable visuellement de la police standard de LaTeX (Computer Modern).
\usepackage[T1]{fontenc}  % Bon encodage des caractères pour qu'Acrobat Reader reconnaisse les accents et les ligatures telles que ffi.
\usepackage[english,frenchb]{babel} % le langage par défaut est le dernier de la liste, c'est-à-dire français
%%
%% Charge le module d'affichage graphique.
\usepackage{graphicx}
\usepackage{epstopdf}  % Permet d'utiliser des .eps avec pdfLaTeX.
%%
%% Recherche des images dans les répertoires.
\graphicspath{{./images/}{./dia/}{./gnuplot/}}
%%
%% Un float peut apparaître seulement après sa définition, jamais avant.
\usepackage{flafter,placeins}
%%
%% Utilisation de natbib pour les citations et la bibliographie.
\usepackage{natbib}
%%
%% Autres packages.
\usepackage{amsmath,color,soulutf8,longtable,colortbl,setspace,ifthen,xspace,url,pdflscape,tikz,pgfplots}
%%
%% Support des acronymes.
\usepackage[nolist]{acronym}
\onehalfspacing                % Interligne 1.5.
%%
%% Définition d'un style de page avec seulement le numéro de page à
%% droite. On s'assure aussi que le style de page par défaut soit
%% d'afficher le numéro de page en haut à droite.
\usepackage{fancyhdr}
\fancypagestyle{pagenumber}{\fancyhf{}\fancyhead[R]{\thepage}}
\renewcommand\headrulewidth{0pt}
\makeatletter
\let\ps@plain=\ps@pagenumber
\makeatother
%%
%% Module qui permet la création des bookmarks dans un fichier PDF.
%\usepackage[dvipdfm]{hyperref}
\usepackage{hyperref}
\usepackage{caption}  % Hyperlien vers la figure plutôt que son titre.

\usepackage{esint}
\usepackage{geometry}
\usepackage{enumerate}

